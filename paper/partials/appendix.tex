\appendix
\section*{Appendices}
\renewcommand{\thesubsection}{\Alph{subsection}}

The following appendices provide additional details that are relevant to the paper. Appendices~\ref{app:jem} and~\ref{app:cp} explain any tasks related to Energy-Based Modelling and Predictive Uncertainty Quantification through Conformal Prediction, respectively. Appendix~\ref{app:eccco} provides additional technical and implementation details about our proposed generator, \textit{ECCCo}, including references to our open-sourced code base. A complete overview of our experimental setup detailing our parameter choices, training procedures and initial black-box model performance can be found in Appendix~\ref{app:setup}. Finally, Appendix~\ref{app:results} reports all of our experimental results in more detail.

\subsection{Energy-Based Modelling}\label{app:jem}

Since we were not able to identify any existing open-source software for Energy-Based Modelling that would be flexible enough to cater to our needs, we have developed a \texttt{Julia} package from scratch. The package has been open-sourced, but to avoid compromising the double-blind review process, we refrain from providing more information at this stage. In our development we have heavily drawn on the existing literature:~\citet{du2019implicit} describe best practices for using EBM for generative modelling;~\citet{grathwohl2020your} explain how EBM can be used to train classifiers jointly for the discriminative and generative tasks. We have used the same package for training and inference, but there are some important differences between the two cases that are worth highlighting here.

\subsubsection{Training: Joint Energy Models}

To train our Joint Energy Models we broadly follow the approach outlined in~\citet{grathwohl2020your}. Formally, JEMs are defined by the following joint distribution:

\begin{equation}
  \begin{aligned}
    \log p_{\theta}(\mathbf{x},\mathbf{y}) &= \log p_{\theta}(\mathbf{y}|\mathbf{x}) + \log p_{\theta}(\mathbf{x})
  \end{aligned}
\end{equation}

Training therefore involves a standard classification loss component $L_{\text{clf}}(\theta)=-\log p_{\theta}(\mathbf{y}|\mathbf{x})$ (e.g. cross-entropy loss) as well as a generative loss component $L_{\text{gen}}(\theta)=-\log p_{\theta}(\mathbf{x})$. Analogous to how we defined the conditional distribution over inputs in Definition~\ref{def:faithful}, $p_{\theta}(\mathbf{x})$ denotes the unconditional distribution over inputs. The model gradient of this component of the loss function can be expressed as follows:

\begin{equation}\label{eq:gen-true}
  \begin{aligned}
    \nabla_{\theta}L_{\text{gen}}(\theta)&=-\nabla_{\theta}\log p_{\theta}(\mathbf{x})=-\left(\mathbb{E}_{p(\mathbf{x})} \left\{  \nabla_{\theta} \mathcal{E}_{\theta}(\mathbf{x}) \right\} - \mathbb{E}_{p_{\theta}(\mathbf{x})} \left\{  \nabla_{\theta} \mathcal{E}_{\theta}(\mathbf{x}) \right\} \right)
  \end{aligned}
\end{equation}

To draw samples from $p_{\theta}(\mathbf{x})$, we rely exclusively on the conditional sampling approach described in~\citet{grathwohl2020your} for both training and inference: we first draw $\mathbf{y}\sim p(\mathbf{y})$ and then sample $\mathbf{x} \sim p_{\theta}(\mathbf{x}|\mathbf{y})$~\citep{grathwohl2020your} via Equation~\ref{eq:sgld} with energy $\mathcal{E}_{\theta}(\mathbf{x}|\mathbf{y})=\mu_{\theta}(\mathbf{x})[\mathbf{y}]$ where $\mu_{\theta}: \mathcal{X} \mapsto \mathbb{R}^K$ returns the linear predictions (logits) of our classifier $M_{\theta}$. While our package also supports unconditional sampling, we found conditional sampling to work well. It is also well aligned with CE, since in this context we are interested in conditioning on the target class. 

As mentioned in the body of the paper, we rely on a biased sampler involving separately specified values for the step size $\epsilon$ and the standard deviation $\sigma$ of the stochastic term involving $\mathbf{r}$. Formally, our biased sampler performs updates as follows: 

\begin{equation}\label{eq:biased-sgld}
  \begin{aligned}
    \hat{\mathbf{x}}_{j+1} &\leftarrow \hat{\mathbf{x}}_j - \frac{\phi}{2} \mathcal{E}_{\theta}(\hat{\mathbf{x}}_j|\mathbf{y}^+) + \sigma \mathbf{r}_j, && j=1,...,J
  \end{aligned}
\end{equation}

Consistent with~\citet{grathwohl2020your}, we have specified $\phi=2$ and $\sigma=0.01$ as the default values for all of our experiments. Here we have deliberately departed slightly from the notation in Equation~\ref{eq:sgld} to emphasize that we use fixed values for $\phi$ and $\sigma$, consistent with the related literature. The number of total SGLD steps $J$ varies by dataset (Table~\ref{tab:ebmparams}). Following best practices, we initialize $\mathbf{x}_0$ randomly in 5\% of all cases and sample from a buffer in all other cases. The buffer itself is randomly initialised and gradually grows to a maximum of 10,000 samples during training as $\hat{\mathbf{x}}_{J}$ is stored in each epoch~\citep{du2019implicit,grathwohl2020your}. 

It is important to realise that sampling is done during each training epoch, which makes training Joint Energy Models significantly harder than conventional neural classifiers. In each epoch the generated (batch of) sample(s) $\hat{\mathbf{x}}_{J}$ is used as part of the generative loss component, which compares its energy to that of observed samples $\mathbf{x}$: 

\begin{equation}\label{eq:gen-loss}
  \begin{aligned}
    L_{\text{gen}}(\theta)&\approx\mu_{\theta}(\mathbf{x})[\mathbf{y}]-\mu_{\theta}(\hat{\mathbf{x}}_{J})[\mathbf{y}]
  \end{aligned}
\end{equation}

Our full training objective can be summarized as follows,

\begin{equation}\label{eq:jem-loss}
  \begin{aligned}
    L_{\text{JEM}}(\theta) &= L_{\text{clf}}(\theta) + L_{\text{gen}}(\theta) + \lambda L_{\text{reg}}(\theta) 
  \end{aligned}
\end{equation}

where $L_{\text{reg}}(\theta)$ is a Ridge penalty (L2 norm) that regularises energy magnitudes for both observed and generated samples~\citep{du2019implicit}. We have used varying degrees of regularization depending on the dataset ($\lambda$ in Table~\ref{tab:ebmparams}). 

Contrary to existing work, we have not typically used the entire minibatch of training data for the generative loss component but found that using a subset of the minibatch was often sufficient in attaining decent generative performance (Table~\ref{tab:ebmparams}). This has helped to reduce the computational burden for our models, which should make it easier for others to reproduce our findings. Figures~\ref{fig:mnist-gen} and~\ref{fig:poc-gen} show generated samples for our \textit{MNIST} and \textit{Moons} data, to provide a sense of their generative property.

\import{../contents/}{table_ebm_params.tex}

\begin{figure}
  \centering
  \includegraphics[width=0.75\linewidth]{../www/mnist_generated_JEM Ensemble.png}
  \caption{Conditionally generated \textit{MNIST} images for our JEM Ensemble.}\label{fig:mnist-gen}
\end{figure}

\begin{figure}
  \centering
  \includegraphics[width=0.5\linewidth]{../www/poc_conditional_samples.png}
  \caption{Conditionally generated samples (stars) for our \textit{Linearly Separable} data using a JEM.}\label{fig:poc-gen}
\end{figure}
\subsubsection{Inference: Quantifying Models' Generative Property}

At inference time, we assume no prior knowledge about the model's generative property. This means that we do not tab into the existing buffer of generated samples for our Joint Energy Models, but instead generate conditional samples from scratch. While we have relied on the default values $\epsilon=2$ and $\sigma=0.01$ also during inference, the number of total SGLD steps was set to $J=500$ in all cases, so significantly higher than during training. For all of our synthetic datasets and models, we generated 50 conditional samples and then formed subsets containing the $n_{E}=25$ lowest-energy samples. While in practice it would be sufficient to do this once for each model and dataset, we have chosen to perform sampling separately for each individual counterfactual in our experiments to account for stochasticity. To help reduce the computational burden for our real-world datasets we have generated only 10 conditional samples each time and used all of them in our counterfactual search. Using more samples, as we originally did, had no substantial impact on our results.

\subsection{Conformal Prediction}\label{app:cp}

In this Appendix~\ref{app:cp} we provide some more background on CP and explain in some more detail how we have used recent advances in Conformal Training for our purposes.

\subsubsection{Background on CP}

Intuitively, CP works under the premise of turning heuristic notions of uncertainty into rigorous uncertainty estimates by repeatedly sifting through the data. It can be used to generate prediction intervals for regression models and prediction sets for classification models. Since the literature on CE and AR is typically concerned with classification problems, we focus on the latter. A particular variant of CP called Split Conformal Prediction (SCP) is well-suited for our purposes, because it imposes only minimal restrictions on model training. 

Specifically, SCP involves splitting the data $\mathcal{D}_n=\{(\mathbf{x}_i,\mathbf{y}_i)\}_{i=1,...,n}$ into a proper training set $\mathcal{D}_{\text{train}}$ and a calibration set $\mathcal{D}_{\text{cal}}$. The former is used to train the classifier in any conventional fashion. The latter is then used to compute so-called nonconformity scores: $\mathcal{S}=\{s(\mathbf{x}_i,\mathbf{y}_i)\}_{i \in \mathcal{D}_{\text{cal}}}$ where $s: (\mathcal{X},\mathcal{Y}) \mapsto \mathbb{R}$ is referred to as \textit{score function}. In the context of classification, a common choice for the score function is just $s_i=1-M_{\theta}(\mathbf{x}_i)[\mathbf{y}_i]$, that is one minus the softmax output corresponding to the observed label $\mathbf{y}_i$~\citep{angelopoulos2021gentle}. 

Finally, classification sets are formed as follows,

\begin{equation}\label{eq:scp}
  \begin{aligned}
    C_{\theta}(\mathbf{x}_i;\alpha)=\{\mathbf{y}: s(\mathbf{x}_i,\mathbf{y}) \le \hat{q}\}
  \end{aligned}
\end{equation}

where $\hat{q}$ denotes the $(1-\alpha)$-quantile of $\mathcal{S}$ and $\alpha$ is a predetermined error rate. As the size of the calibration set increases, the probability that the classification set $C(\mathbf{x}_{\text{test}})$ for a newly arrived sample $\mathbf{x}_{\text{test}}$ does not cover the true test label $\mathbf{y}_{\text{test}}$ approaches $\alpha$~\citep{angelopoulos2021gentle}. 

Observe from Equation~\ref{eq:scp} that Conformal Prediction works on an instance-level basis, much like CE are local. The prediction set for an individual instance $\mathbf{x}_i$ depends only on the characteristics of that sample and the specified error rate. Intuitively, the set is more likely to include multiple labels for samples that are difficult to classify, so the set size is indicative of predictive uncertainty. To see why this effect is exacerbated by small choices for $\alpha$ consider the case of $\alpha=0$, which requires that the true label is covered by the prediction set with probability equal to 1.

\subsubsection{Differentiability}\label{app:cp-diff}

The fact that conformal classifiers produce set-valued predictions introduces a challenge: it is not immediately obvious how to use such classifiers in the context of gradient-based counterfactual search. Put differently, it is not clear how to use prediction sets in Equation~\ref{eq:general}. Fortunately, \citet{stutz2022learning} have recently proposed a framework for Conformal Training that also hinges on differentiability. Specifically, they show how Stochastic Gradient Descent can be used to train classifiers not only for the discriminative task but also for additional objectives related to Conformal Prediction. One such objective is \textit{efficiency}: for a given target error rate $\alpha$, the efficiency of a conformal classifier improves as its average prediction set size decreases. To this end, the authors introduce a smooth set size penalty defined in Equation~\ref{eq:setsize} in the body of this paper. Formally, it is defined as $C_{\theta,\mathbf{y}}(\mathbf{x}_i;\alpha):=\sigma\left((s(\mathbf{x}_i,\mathbf{y})-\alpha) T^{-1}\right)$ for $\mathbf{y}\in\mathcal{Y}$, where $\sigma$ is the sigmoid function and $T$ is a hyper-parameter used for temperature scaling~\citep{stutz2022learning}.

In addition to the smooth set size penalty,~\citet{stutz2022learning} also propose a configurable classification loss function, that can be used to enforce coverage. For \textit{MNIST} data, we found that using this function generally improved the visual quality of the generated counterfactuals, so we used it in our experiments involving real-world data. For the synthetic dataset, visual inspection of the counterfactuals showed that using the configurable loss function sometimes led to overshooting: counterfactuals would end up deep inside the target domain but far away from the observed samples. For this reason, we instead relied on standard cross-entropy loss for our synthetic datasets. As we have noted in the body of the paper, more experimental work is certainly needed in this context. Figure~\ref{fig:cp-diff} shows the prediction set size (left), smooth set size loss (centre) and configurable classification loss (right) for a \textit{JEM} trained on our \textit{Linearly Separable} data.

\begin{figure}
  \centering
  \includegraphics[width=1.0\linewidth]{../www/poc_set_size.png}
  \caption{Prediction set size (left), smooth set size loss (centre) and configurable classification loss (right) for a JEM trained on our \textit{Linearly Separable} data.}\label{fig:cp-diff}
\end{figure}

\subsection{\textit{ECCCo}}\label{app:eccco}

In this section, we explain \textit{ECCCo} in some more detail, briefly discuss convergence conditions for counterfactual explanations and provide details concerning the actual implementation of our framework in \texttt{Julia}.  

\subsubsection{Deriving the search objective} 

The counterfactual search objective for \textit{ECCCo} was introduced in Equation~\ref{eq:eccco} in the body of the paper. We restate this equation here for reference:

\begin{equation} \label{eq:eccco-app}
  \begin{aligned}
  \mathbf{Z}^\prime &= \arg \min_{\mathbf{Z}^\prime \in \mathcal{Z}^L} \{  {\text{yloss}(M_{\theta}(f(\mathbf{Z}^\prime)),\mathbf{y}^+)}+ \lambda_{1} {\text{dist}(f(\mathbf{Z}^\prime),\mathbf{x}) } \\
  &+ \lambda_2 \mathcal{E}_{\theta}(\mathbf{Z}^\prime,\widehat{\mathbf{X}}_{\theta,\mathbf{y}^+}) + \lambda_3 \Omega(C_{\theta}(f(\mathbf{Z}^\prime);\alpha)) \} 
  \end{aligned} 
\end{equation}

We can make the connection to energy-based modeling more explicit by restating the counterfactual search objective in terms $L_{\text{JEM}}(\theta)$, which we defined in Equation~\ref{eq:jem-loss}. In particular, consider the following counterfactual search objective,

\begin{equation} \label{eq:eccco-jem}
  \begin{aligned}
  \mathbf{Z}^\prime &= \arg \min_{\mathbf{Z}^\prime \in \mathcal{Z}^L} \{  {L_{\text{JEM}}(\theta;M_{\theta}(f(\mathbf{Z}^\prime)),\mathbf{y}^+)}+ \lambda_{1} {\text{dist}(f(\mathbf{Z}^\prime),\mathbf{x}) }  + \lambda_3 \Omega(C_{\theta}(f(\mathbf{Z}^\prime);\alpha)) \} 
  \end{aligned} 
\end{equation}

where we have simply used the JEM loss function as $\text{yloss}(M_{\theta}(f(\mathbf{Z}^\prime)),\mathbf{y}^+)$.

Now note that aside from the additional penalties in Equation~\ref{eq:eccco-app}, the only key difference between our counterfactual search objective and the joint-energy training objective is the parameter that is being optimized. In joint-energy training we optimize the objective with respect to the network weights $\theta$. Recall that $\mathcal{E}_{\theta}(\mathbf{x}|\mathbf{y})=\mu_{\theta}(\mathbf{x})[\mathbf{y}]$. Then the partial gradient with respect to the generative loss component of $L_{\text{JEM}}(\theta)$ can be expressed as follows:

\begin{equation}\label{eq:jem-grad}
  \begin{aligned}
    \nabla_{\theta}L_{\text{gen}}(\theta) &= \nabla_{\theta}\mu_{\theta}(\mathbf{x})[\mathbf{y}]- \nabla_{\theta}\mu_{\theta}(\hat{\mathbf{x}}_{J})[\mathbf{y}]
  \end{aligned}
\end{equation}

During the counterfactual search, we take the network parameters as fixed and instead optimize with respect to the counterfactual itself\footnote{Here we omit the notion of a latent search space to make the comparison easier.},

\begin{equation}\label{eq:ce-grad}
  \begin{aligned}
    \nabla_{\mathbf{x}}L_{\text{gen}}(\theta) &= \nabla_{\mathbf{x}}\mu_{\theta}(\mathbf{x})[\mathbf{y}^+]- \nabla_{\mathbf{x}}\mu_{\theta}(\hat{\mathbf{x}}_{J})[\mathbf{y}^+]=\nabla_{\mathbf{x}}\mu_{\theta}(\mathbf{x})[\mathbf{y}^+]=\nabla_{\mathbf{x}}\mathcal{E}_{\theta}(\mathbf{x}|\mathbf{y}^+)
  \end{aligned}
\end{equation}

where the second term is equal to zero because $\mu_{\theta}(\hat{\mathbf{x}}_{J})[\mathbf{y}]$ is invariant with respect to $\mathbf{x}$. Since this term has zero gradients, we can remove it from the loss function altogether. For the regularization loss component of $L_{\text{JEM}}(\theta)$ we can proceed analogously such that we can rewrite Equation~\ref{eq:eccco-jem} as follows:

\begin{equation} \label{eq:eccco-jem-2}
  \begin{aligned}
  \mathbf{Z}^\prime =& \arg \min_{\mathbf{Z}^\prime \in \mathcal{Z}^L} \{  {\text{yloss}(M_{\theta}(f(\mathbf{Z}^\prime)),\mathbf{y}^+) + \mathcal{E}_{\theta}(f(\mathbf{Z}^\prime)|\mathbf{y}^+) + || \mathcal{E}_{\theta}(f(\mathbf{Z}^\prime)|\mathbf{y}^+) ||_2^2} \\ &+ \lambda_{1} {\text{dist}(f(\mathbf{Z}^\prime),\mathbf{x}) }  + \lambda_3 \Omega(C_{\theta}(f(\mathbf{Z}^\prime);\alpha)) \} 
  \end{aligned} 
\end{equation}

Now we notice that Equation~\ref{eq:eccco-jem-2} is equivalent to Equation~\ref{eq:eccco-app} for $\lambda_2=1$. For the sake of simplicity, we omitted the regularization component from Equation~\ref{eq:eccco} in the main text. Intuitively, taking iterative gradient steps according to Equation~\ref{eq:ce-grad} has the effect of constraining the energy of the counterfactual until. The generative property of the underlying model implicitly enters this equation through $\theta$.

\subsubsection{The \textit{ECCCo} algorithm}

Algorithm~\ref{alg:eccco} describes how exactly \textit{ECCCo} works. For the sake of simplicity and without loss of generality, we limit our attention to generating a single counterfactual $\mathbf{x}^\prime=f(\mathbf{z}^\prime)$. The counterfactual state $\mathbf{z}^\prime$ is initialized at the factual $\mathbf{x}$. Other forms of initialization are also suitable but not considered here. For example, one may choose at a small random perturbation to all features~\citep{slack2021counterfactual}. Next, we calibrate the model $M_{\theta}$ through split conformal prediction. Finally, we search counterfactuals through gradient descent where $\mathcal{L}(\mathbf{z}^\prime,\mathbf{y}^+,\widehat{\mathbf{X}}_{\theta,\mathbf{y}^+}; \Lambda, \alpha)$ denotes our loss function defined in Equation~\ref{eq:eccco}. The search terminates once the convergence criterium is met or the maximum number of iterations $T$ has been exhausted. Note that the choice of convergence criterium has important implications on the final counterfactual which we explain below.

\begin{algorithm*}[h]
  \caption{The \textit{ECCCo} generator}\label{alg:eccco}
  \begin{algorithmic}[1]
    \Require $\mathbf{x}, \mathbf{y}^+, M_{\theta}, \Lambda=[\lambda_1,\lambda_2,\lambda_3], \alpha, \mathcal{D}, T$ where $M_{\theta}(\mathbf{x})\neq\mathbf{y}^+$
    \Ensure $\mathbf{x}^\prime$
    \State Initialize $\mathbf{z}^\prime \gets \mathbf{x}$ 
    \State Run \textit{SCP} for $M_{\theta}$ using $\mathcal{D}$ \Comment{Calibrate model through split conformal prediction.}
    \State Initialize $t \gets 0$
    \While{\textit{not converged} or $t < T$} \Comment{For convergence conditions see below.}
    \State $\mathbf{z}^\prime \gets \mathbf{z}^\prime - \eta \nabla_{\mathbf{z}^\prime} \mathcal{L}(\mathbf{z}^\prime,\mathbf{y}^+; \Lambda, \alpha)$ \Comment{Take gradient step of size $\eta$.}
    \State $t \gets t+1$
    \EndWhile
    \State $\mathbf{x}^\prime \gets \mathbf{z}^\prime$
  \end{algorithmic}
\end{algorithm*}

\subsubsection{The \textit{ECCCo+} algorithm}

Algorithm~\ref{alg:eccco-plus} describes how exactly \textit{ECCCo+} works. The only difference to \textit{ECCCo} is that we encode and decode features using PCA. In particular, we let $f^{-1}(\mathbf{x})$ denote the projection of $\mathbf{x}$ to its first $n_z$ principal components. Conversely, $f(\mathbf{z}^\prime)$ maps back from the projection to the feature space. 

\begin{algorithm*}[h]
  \caption{The \textit{ECCCo+} generator}\label{alg:eccco-plus}
  \begin{algorithmic}[1]
    \Require $\mathbf{x}, \mathbf{y}^+, M_{\theta}, f, \Lambda=[\lambda_1,\lambda_2,\lambda_3], \alpha, \mathcal{D}, T$ where $M_{\theta}(\mathbf{x})\neq\mathbf{y}^+$
    \Ensure $\mathbf{x}^\prime$
    \State Initialize $\mathbf{z}^\prime \gets f^{-1}(\mathbf{x})$ \Comment{Map to counterfactual state space.}
    \State Run \textit{SCP} for $M_{\theta}$ using $\mathcal{D}$ \Comment{Calibrate model through split conformal prediction.}
    \State Initialize $t \gets 0$
    \While{\textit{not converged} or $t < T$} \Comment{For convergence conditions see below.}
    \State $\mathbf{z}^\prime \gets \mathbf{z}^\prime - \eta \nabla_{\mathbf{z}^\prime} \mathcal{L}(\mathbf{z}^\prime,\mathbf{y}^+; \Lambda, \alpha)$ \Comment{Take gradient step of size $\eta$.}
    \State $t \gets t+1$
    \EndWhile
    \State $\mathbf{x}^\prime \gets f(\mathbf{z}^\prime)$ \Comment{Map back to feature space.}
  \end{algorithmic}
\end{algorithm*}

\subsubsection{The \textit{ECCCo-L1} algorithm}

Algorithm~\ref{alg:eccco-l1} describes a variation of \textit{ECCCo} that we initally considered but ultimately discarded. For the sake of completeness we have included this approach here in the appendix. It generally yields very faithful counterfactuals but it is computationally much more expensive and struggles with plausibility.

Instead of constraining energy directly, this approach works under the premise of penalizing the distance between the counterfactual and samples generated through SGLD. The counterfactual state $\mathbf{z}^\prime$ is initialized as in Algorithm~\ref{alg:eccco}. Next, we generate $n_{\mathcal{B}}$ conditional samples $\hat{\mathbf{x}}_{\theta,\mathbf{y}^+}$ using SGLD (Equation~\ref{eq:sgld}) and store the $n_E$ instances with the lowest energy. We then calibrate the model $M_{\theta}$ through split conformal prediction. Finally, we search counterfactuals through gradient descent where $\mathcal{L}(\mathbf{z}^\prime,\mathbf{y}^+,\widehat{\mathbf{X}}_{\theta,\mathbf{y}^+}; \Lambda, \alpha)$ denotes our loss function defined in Equation~\ref{eq:eccco}, but instead of constraining energy directly we use Equaqtion~\ref{eq:faith} (unfaithfulness metric) as a penalty.

\begin{algorithm*}[h]
  \caption{The \textit{ECCCo-L1} generator}\label{alg:eccco-l1}
  \begin{algorithmic}[1]
    \Require $\mathbf{x}, \mathbf{y}^+, M_{\theta}, f, \Lambda=[\lambda_1,\lambda_2,\lambda_3], \alpha, \mathcal{D}, T, \eta, n_{\mathcal{B}}, n_E$ where $M_{\theta}(\mathbf{x})\neq\mathbf{y}^+$
    \Ensure $\mathbf{x}^\prime$
    \State Initialize $\mathbf{z}^\prime \gets \mathbf{x}$ 
    \State Generate $\left\{\hat{\mathbf{x}}_{\theta,\mathbf{y}^+}\right\}_{n_{\mathcal{B}}} \gets p_{\theta}(\mathbf{x}_{\mathbf{y}^+})$ \Comment{Generate $n_{\mathcal{B}}$ samples using SGLD (Equation~\ref{eq:sgld}).}
    \State Store $\widehat{\mathbf{X}}_{\theta,\mathbf{y}^+} \gets \left\{\hat{\mathbf{x}}_{\theta,\mathbf{y}^+}\right\}_{n_{\mathcal{B}}}$ \Comment{Choose $n_E$ lowest-energy samples.}
    \State Run \textit{SCP} for $M_{\theta}$ using $\mathcal{D}$ \Comment{Calibrate model through split conformal prediction.}
    \State Initialize $t \gets 0$
    \While{\textit{not converged} or $t < T$} \Comment{For convergence conditions see below.}
    \State $\mathbf{z}^\prime \gets \mathbf{z}^\prime - \eta \nabla_{\mathbf{z}^\prime} \mathcal{L}(\mathbf{z}^\prime,\mathbf{y}^+,\widehat{\mathbf{X}}_{\theta,\mathbf{y}^+}; \Lambda, \alpha)$ \Comment{Take gradient step of size $\eta$.}
    \State $t \gets t+1$
    \EndWhile
    \State $\mathbf{x}^\prime \gets \mathbf{z}^\prime$
  \end{algorithmic}
\end{algorithm*}

\subsubsection{A Note on Convergence}\label{convergence}

Convergence is not typically discussed much in the context of CE, even though it has important implications on outcomes. One intuitive way to specify convergence is in terms of threshold probabilities: once the predicted probability $p(\mathbf{y}^+|\mathbf{x}^{\prime})$ exceeds some user-defined threshold $\gamma$ such that the counterfactual is valid, we could consider the search to have converged. In the binary case, for example, convergence could be defined as $p(\mathbf{y}^+|\mathbf{x}^{\prime})>0.5$ in this sense. Note, however, how this can be expected to yield counterfactuals in the proximity of the decision boundary, a region characterized by high aleatoric uncertainty. In other words, counterfactuals generated in this way would generally not be plausible. To avoid this from happening, we specify convergence in terms of gradients approaching zero for all our experiments and all of our generators. This is allows us to get a cleaner read on how the different counterfactual search objectives affect counterfactual outcomes. 

\subsubsection{\texttt{ECCCo.jl}}

We have built a small Julia package, \texttt{ECCCo.jl}, that was used to perform the analysis in this paper: https://github.com/pat-alt/ECCCo.jl. The package will be absorbed by \texttt{CounterfactualExplanations.jl} in the future~\citep{altmeyer2023explaining}.

\subsection{Experimental Setup}\label{app:setup}

In our experiments, we always generate multiple counterfactuals for each model and generator over multiple runs to account for stochasticity. For each generator, model and dataset, we sample a total of $n_f=100$ factuals for a single benchmark, where each time the factual and target class is drawn randomly. This process is repeated over 50 runs for the synthetic and tabular datasets, which yields a total of around 1.5 million counterfactuals per dataset for the final results. For vision datasets, we relied on 10 runs yielding a total of 300,000 counterfactuals. The final results presented in the paper and this appendix show the sample averages and the standard deviations of samples averages across all runs. For grid search, we used single runs for each parameter specification generating 100 counterfactuals per generator and model for synthetic and tabular datasets. For vision data, we generated 10 counterfactuals for each model, generator and parameter specification.

Table~\ref{tab:params} provides an overview of all parameters related to our experiments. The \textit{GMSC} data were randomly undersampled for balancing purposes and all features were standardized. \textit{MNIST} data was also randomly undersampled for reasons outlined below. Pixel values were preprocessed to fall in the range of $[-1,1]$ and a small Gaussian noise component ($\sigma=0.03$) was added to training samples following common practice in the EBM literature. All of our models were trained through mini-batch training using the Adam optimiser (\citet{kingma2014adam}). Table~\ref{tab:perf} shows standard evaluation metrics measuring the predictive performance of our different models grouped by dataset. These measures were computed on test data. 

Table~\ref{tab:genparams} summarises our hyperparameter choices for the counterfactual generators where $\eta$ denotes the learning rate used for Stochastic Gradient Descent (SGD) and $\lambda_1$, $\lambda_2$, $\lambda_3$ represent the chosen penalty strengths (Equations~\ref{eq:general} and~\ref{eq:eccco}). Here $\lambda_1$ also refers to the chosen penalty for the distance from factual values that applies to both \textit{Wachter} and \textit{REVISE}, but not \textit{Schut} which is penalty-free. \textit{Schut} is also the only generator that uses JSMA instead of SGD for optimization.

\import{../contents/}{table_params.tex}

\import{../contents/}{table_perf.tex}

\import{../contents/}{table_gen_params.tex}

\subsection{Compute}

Research reported in this work was partially or completely facilitated by computational resources and support of the DelftBlue~\citep{DHPC2022} and the Delft AI Cluster (DAIC: https://doc.daic.tudelft.nl/) at TU Delft. 

For grid search, we used 300 CPUs for tabular real-world datasets ($<$1.5 hours each), 150 CPUs for synthetic datasets ($\approx$1 hour each) and 100 CPUs for vision datasets ($<$4 hours each), where we ran a smaller grid search for the latter. To generate the final results reported in the tables we used 300 CPUs for all datasets ($<$1.5 hours each) except the vision datasets. For the latter, we used 50 CPUs for smaller final experiments and at longer run times ($\approx$5 hours each).

\subsection{Results}\label{app:results}

Figures~\ref{fig:mnist-eccco-lenet} to~\ref{fig:mnist-eccco-jem-ens} show examples of counterfactuals for \textit{MNIST} generated by \textit{ECCCo+} for our different models. Original images are shown on the diagonal and the corresponding counterfactuals are plotted across rows. Figures~\ref{fig:mnist-revise-lenet} to~\ref{fig:mnist-revise-jem-ens} show the same examples but for \textit{REVISE}. Both counterfactual generators have access to the same optimizer. While the results for \textit{REVISE} look fairly poor here, we have observed better results for optizers with higher step sizes. Note that the seemingly poor performance by \textit{REVISE} upon visual inspection is not driven by a weak surrogate VAE: Figure~\ref{fig:vae-rec} shows image reconstructions generated by the VAE.

% ECCCo params λ=[0.1,0.5,1.0], reg_strength=0.0

\begin{figure}
  \centering
  \includegraphics[width=1.0\linewidth]{../www/mnist_all_lenet_eccco.png}
  \caption{Counterfactuals for \textit{MNIST} data generated by \textit{ECCCo+}. The underlying model is a LeNet-5 \textit{CNN}. Original images are shown on the diagonal with the corresponding counterfactuals plotted across rows.}\label{fig:mnist-eccco-lenet}
\end{figure}

\begin{figure}
  \centering
  \includegraphics[width=1.0\linewidth]{../www/mnist_all_mlp_eccco.png}
  \caption{Counterfactuals for \textit{MNIST} data generated by \textit{ECCCo+}. The underlying model is an \textit{MLP}. Original images are shown on the diagonal with the corresponding counterfactuals plotted across rows.}\label{fig:mnist-eccco-mlp}
\end{figure}

\begin{figure}
  \centering
  \includegraphics[width=1.0\linewidth]{../www/mnist_all_mlp_ens_eccco.png}
  \caption{Counterfactuals for \textit{MNIST} data generated by \textit{ECCCo+}. The underlying model is an \textit{MLP} ensemble. Original images are shown on the diagonal with the corresponding counterfactuals plotted across rows.}\label{fig:mnist-eccco-mlp-ens}
\end{figure}

\begin{figure}
  \centering
  \includegraphics[width=1.0\linewidth]{../www/mnist_all_jem_eccco.png}
  \caption{Counterfactuals for \textit{MNIST} data generated by \textit{ECCCo+}. The underlying model is a \textit{JEM}. Original images are shown on the diagonal with the corresponding counterfactuals plotted across rows.}\label{fig:mnist-eccco-jem}
\end{figure}

\begin{figure}
  \centering
  \includegraphics[width=1.0\linewidth]{../www/mnist_all_jem_ens_eccco.png}
  \caption{Counterfactuals for \textit{MNIST} data generated by \textit{ECCCo+}. The underlying model is a \textit{JEM} ensemble. Original images are shown on the diagonal with the corresponding counterfactuals plotted across rows.}\label{fig:mnist-eccco-jem-ens}
\end{figure}

% REVISE

\begin{figure}
  \centering
  \includegraphics[width=1.0\linewidth]{../www/mnist_all_lenet_revise.png}
  \caption{Counterfactuals for \textit{MNIST} data generated by \textit{REVISE}. The underlying model is a LeNet-5 \textit{CNN}. Original images are shown on the diagonal with the corresponding counterfactuals plotted across rows.}\label{fig:mnist-revise-lenet}
\end{figure}

\begin{figure}
  \centering
  \includegraphics[width=1.0\linewidth]{../www/mnist_all_mlp_revise.png}
  \caption{Counterfactuals for \textit{MNIST} data generated by \textit{REVISE}. The underlying model is an \textit{MLP}. Original images are shown on the diagonal with the corresponding counterfactuals plotted across rows.}\label{fig:mnist-revise-mlp}
\end{figure}

\begin{figure}
  \centering
  \includegraphics[width=1.0\linewidth]{../www/mnist_all_mlp_ens_revise.png}
  \caption{Counterfactuals for \textit{MNIST} data generated by \textit{REVISE}. The underlying model is an \textit{MLP} ensemble. Original images are shown on the diagonal with the corresponding counterfactuals plotted across rows.}\label{fig:mnist-revise-mlp-ens}
\end{figure}

\begin{figure}
  \centering
  \includegraphics[width=1.0\linewidth]{../www/mnist_all_jem_revise.png}
  \caption{Counterfactuals for \textit{MNIST} data generated by \textit{REVISE}. The underlying model is a \textit{JEM}. Original images are shown on the diagonal with the corresponding counterfactuals plotted across rows.}\label{fig:mnist-revise-jem}
\end{figure}

\begin{figure}
  \centering
  \includegraphics[width=1.0\linewidth]{../www/mnist_all_jem_ens_revise.png}
  \caption{Counterfactuals for \textit{MNIST} data generated by \textit{REVISE}. The underlying model is a \textit{JEM} ensemble. Original images are shown on the diagonal with the corresponding counterfactuals plotted across rows.}\label{fig:mnist-revise-jem-ens}
\end{figure}

\begin{figure}[h]
  \centering
  \includegraphics[width=1.0\linewidth]{../www/vae_rec.png}
  \caption{Randomly drawn \textit{MNIST} images and their reconstructions generated by the VAE used by \textit{REVISE}.}\label{fig:vae-rec}
\end{figure}

Tables~\ref{tab:results-linearly-separable} to~\ref{tab:results-fashion-mnist} reports all of the evaluation metrics we have computed. Tables~\ref{tab:results-linearly-separable-valid} to~\ref{tab:results-fashion-mnist-valid} reports the same metrics for the subset of valid counterfactuals. The `Unfaithfulness' and `Implausibility' metrics have been discussed extensively in the body of the paper. The `Cost' metric relates to the distance between the factual and the counterfactual and is measured using the L1 Norm. The `Redundancy' metric measures sparsity in is defined as the percentage of features that remain unperturbed (higher is better). The `Uncertainty' metric is just the average value of the smooth set size penalty (Equation~\ref{eq:setsize}). Finally, `Validity' is the percentage of valid counterfactuals. 

\import{../contents/}{table-linearly-separable.tex}

\import{../contents/}{table-circles.tex}

\import{../contents/}{table-moons.tex}

\import{../contents/}{table-california-housing.tex}

\import{../contents/}{table-gmsc.tex}

\import{../contents/}{table-german-credit.tex}

\import{../contents/}{table-mnist.tex}

\import{../contents/}{table-fashion-mnist.tex}

\import{../contents/}{table-linearly-separable-valid.tex}

\import{../contents/}{table-circles-valid.tex}

\import{../contents/}{table-moons-valid.tex}

\import{../contents/}{table-california-housing-valid.tex}

\import{../contents/}{table-gmsc-valid.tex}

\import{../contents/}{table-german-credit-valid.tex}

\import{../contents/}{table-mnist-valid.tex}

\import{../contents/}{table-fashion-mnist-valid.tex}